\documentclass[11pt]{article}
%% Escrevendo em português
\usepackage[brazil]{babel}
\usepackage[utf8]{inputenc}
%\usepackage[latin1]{inputenc}
\usepackage[usenames,dvipsnames,svgnames,table]{xcolor}
\usepackage[a4paper,margin={1in}]{geometry}
\usepackage{graphicx}
\usepackage{color}
\usepackage{tikz}
\usepackage{mathtools}
\usepackage{enumitem}
\definecolor{warning}{rgb}{0.8, 0, 0}

\renewcommand{\baselinestretch}{1.5}
\newcommand{\vsp}{\vspace{0.2in}}

\begin {document}


\centerline{
  \begin{minipage}[t]{5in}
    \begin{center}
    {\Large \bf RELATÓRIO - EP3}
    \vsp \\
  {\large Renan Fichberg - {\bf NUSP:} 7991131}\\
	{\small Laboratório de Métodos Numéricos - MAC0210 - 2017/1}\\
	{\small {\bf Professor:} Ernesto G. Birgin}\\
  {\small {\bf Monitor:} Lucas Magno}
    \end{center}
  \end{minipage}
}
\vsp

%============================================================
\pagebreak

\section{Arquivos e diretórios}

\indent\indent Neste exercício programa, estão sendo entregues os seguintes arquivos e diretórios:

\begin{itemize}
  \item {\ttfamily{/docs}} - Diretório contendo este relatório.
  \item {\ttfamily{/docs/relatorio.pdf}} - Este documento.
  \item {\ttfamily{/images}} - Diretório que contém a imagem original e as geradas (se for rodar o programa que manipula imagens ao invés de funções).
  \item {\ttfamily{/images/purple\_tentacle.jpg}} - Imagem original. Está sendo entregue como exemplo de entrada do programa de imagens para efeitos ilustrativos.
  \item {\ttfamily{/images/compressed\_red.jpg}} - Imagem gerada pela compressão da imagem {\ttfamily{purple\_tentacle.jpg}}. Está sendo entregue (e o programa a escreve no disco) como exemplo de saída do programa para efeitos ilustrativos.
  \item {\ttfamily{/images/compressed\_blue.jpg}} - Idem a imagem {\ttfamily{compressed\_red.jpg}}.
  \item {\ttfamily{/images/compressed\_green.jpg}} - Idem as imagens {\ttfamily{compressed\_red.jpg}} e {\ttfamily{compressed\_blue.jpg}}.
  \item {\ttfamily{/images/decompressed\_red.jpg}} - Imagem gerada pela decompressão da imagem {\ttfamily{decompressed\_red.jpg}}. Está sendo entregue como exemplo de saída do programa para efeitos ilustrativos.
  Retorna a imagem ao estado anterior (isto é, restaura a imagem {\ttfamily{compressed\_red.jpg}} através dos métodos de interpolação bicúbica ou bilinear nos \textit{pixels} de {\ttfamily{compressed.jpg}}), com perdas.
  \item {\ttfamily{/images/decompressed\_blue.jpg}} - Idem a imagem {\ttfamily{decompressed\_red.jpg}}.
  \item {\ttfamily{/images/decompressed\_green.jpg}} - Idem as imagens {\ttfamily{decompressed\_red.jpg}} e {\ttfamily{decompressed\_blue.jpg}}.
  \item {\ttfamily{/images/final.jpg}} - Junção dos canais para formar a imagem original (a qualidade está ruim por causa de um problema com a biblioteca {\ttfamily{GraphicsMagick}}).
  \item {\ttfamily{/src}} - Diretório contendo os códigos-fonte do Exercício Programa 3.
  \item {\ttfamily{/src/argument\_checker.m}} - Código-fonte do Exercício Programa 3 para leitura dos argumentos da CLI, chamado pelo \textit{script} principal selecionado (um dos dois listados nos itens abaixo).
  \item {\ttfamily{/src/bivariate\_interpolation.m}} - Código-fonte do Exercício Programa 3 para manipulação de uma \textit{imagem}.
  \item {\ttfamily{/src/bivariate\_interpolation\_test.m}} - Código-fonte do Exercício Programa 3 para manipulação de uma \textit{função conhecida f(x, y)}.
\end{itemize}

\pagebreak

\section{Invocação dos Programas}

\indent\indent O exercício programa foi dividido em dois \textit{scripts} devido a minha experiência obtida com o EP2, na qual eu achei que o código acabou tendo muitas condições de desvio
que acabaram aumentando a complexidade da função principal, tornando-o difícil (e chato) de ler. Para resolver este problema, achei melhor desenvolver dois \textit{scripts} independentes. O código-fonte
de ambos difere de muita pouca coisa, sendo que o que foi pedido tanto no EP2 quanto no EP3 está implementado da mesma forma em ambos os \textit{scripts}. Todas as funções implementadas, porém, são encontradas
em ambos os códigos-fonte com a mesma assinatura, diferindo apenas nos parâmetros (no de imagem, os parâmetros referentes às malhas são triplicados por causa dos três canais de cores. Idem eventuais retornos das malhas
pelas funções).

\subsection{Manipulação de imagens}

\indent\indent Para rodar o programa no modo imagem, é necessário estar no diretório do código-fonte `{\ttfamily{/src}}' e usar o seguinte comando:

\begin{flushleft}
  {\ttfamily{\$ ./bivariate\_interpolation.m <parametros>}}
\end{flushleft}

Onde os {\ttfamily{<parâmetros>}} disponíveis são os descritos abaixo (todos antecedidos de \textit{dois} caracters ``-''):


\begin{itemize}
  \item \textbf{--image}: parâmetro \textbf{mandatório}. Deve ser uma \textbf{imagem}. Os testes foram realizados na imagem que está sendo fornecida, com a extensão {\ttfamily{.jpg}}. A
  imagem deve \textbf{obrigatoriamente} estar no diretório `{\ttfamily{/images}}'.
  \item \textbf{--bilinear}: parâmetro \textbf{semi-mandatório}. Passe este parâmetro, sem argumentos, se quiser rodar o programa no modo de interpolação bilinear. Este
  parâmetro está classificado como ``semi-mandatório'' pois ele não é obrigado estar presente, desde que o parâmetro para o modo bicúbico esteja. \textbf{Um dos dois parâmetros de modo
  deve ser obrigatoriamente passado}.
  \item \textbf{--bicubic}: parâmetro \textbf{semi-mandatório}. Idem `{\ttfamily{--bilinear}}', só que para rodar o programa no modo bicúbico. Apenas um dos parâmetros `semi-mandatórios' pode ser passado por vez.
  \item \textbf{--cr}: parâmetro \textbf{mandatório}. Deve ser um \textbf{número inteiro}. É a taxa de compressão da imagem (\textbf{c}ompression \textbf{r}ate). Nos testes o valor mais usado foi `5'.
\end{itemize}

\pagebreak


Exemplo válido de invocação do \textit{script} de imagem:
\begin{flushleft}
  {\ttfamily{\$ ./bivariate\_interpolation.m --image purple\_tentacle.jpg --bicubic --cr 5}}
\end{flushleft}

\subsection{Manipulação de funções e rotinas de teste}

\indent\indent Para rodar o programa no modo de funções, é necessário estar no diretório do código-fonte `{\ttfamily{/src}}' e usar o seguinte comando:

\begin{flushleft}
  {\ttfamily{\$ ./bivariate\_interpolation\_test.m <parametros>}}
\end{flushleft}

Onde os {\ttfamily{<parâmetros>}} disponíveis são os descritos abaixo (todos antecedidos de \textit{dois} caracters ``-''):

\begin{itemize}
  \item \textbf{--nx}: parâmetro \textbf{mandatório}. Deve ser um \textbf{número inteiro e positivo}, conforme especificado no enunciado.
  \item \textbf{--ny}: parâmetro \textbf{mandatório}. Deve ser um \textbf{número inteiro e positivo}, conforme especificado no enunciado.
  \item \textbf{--ax}: parâmetro \textbf{mandatório}. Deve ser um \textbf{número real}, conforme especificado no enunciado.
  \item \textbf{--bx}: parâmetro \textbf{mandatório}. Deve ser um \textbf{número real}, conforme especificado no enunciado. Ainda, deve satisfazer a condição $a_x < b_x$.
  \item \textbf{--ay}: parâmetro \textbf{mandatório}. Deve ser um \textbf{número real}, conforme especificado no enunciado.
  \item \textbf{--by}: parâmetro \textbf{mandatório}. Deve ser um \textbf{número real}, conforme especificado no enunciado. Ainda, deve satisfazer a condição $a_y < b_y$.
  \item \textbf{--x}: parâmetro \textbf{mandatório}. Deve ser um \textbf{número real contido na malha}.
  \item \textbf{--y}: parâmetro \textbf{mandatório}. Deve ser um \textbf{número real contido na malha}.
  \item \textbf{--bilinear}: parâmetro \textbf{semi-mandatório}. Passe este parâmetro, sem argumentos, se quiser rodar o programa no modo de interpolação bilinear. Este
  parâmetro está classificado como ``semi-mandatório'' pois ele não é obrigado estar presente, desde que o parâmetro para o modo bicúbico esteja. \textbf{Um dos dois parâmetros de modo
  deve ser obrigatoriamente passado}.
  \item \textbf{--bicubic}: parâmetro \textbf{semi-mandatório}. Idem `{\ttfamily{--bilinear}}', só que para rodar o programa no modo bicúbico. Apenas um dos parâmetros `semi-mandatórios' pode ser passado por vez.
\end{itemize}

Exemplo válido de invocação do \textit{script} de funções:
\begin{flushleft}
  {\ttfamily{\$ ./bivariate\_interpolation\_test.m --nx 150 --ny 150 --ax -5 --bx 5 --ay -5 --by 5 --x 3.11 --y 1.42 --bicubic}}
\end{flushleft}

\subsubsection{Testes realizados neste modo}

\begin{enumerate}
  \item Teste de verificação: o programa irá tentar mostrar ao usuário que, para os valores de parâmetros escolhidos, $v$ interpola $f$ nos pontos da malha. Ou seja, será mostrado
  que para todos os pontos $(x, y)$ da malha, $|f(x, y) - v(x, y)| = 0$.

  \textbf{Nota importante:} Aparentemente foi encontrado um bug no \textit{GNU-Octave}. O exemplo fornecido abaixo para rodar
  o programa neste modo o reproduz. O teste ira acusar como falho, mas os valores de `a' e `b' são iguais, conforme pode ser evidenciado ao remover o comentário do \textit{printf} da função `interpolation\_verification\_test'.
  \item Teste de comportamento do erro: se o programa detectar que a área é quadrada, isto é, $h_x = h_y$, serão \textit{duplicados} os valores dos parâmetros $n_x$ e $n_y$ de forma a mostrar
  que o erro diminui com o aumento da quantidade de pontos na malha fina (em outras palavras, quando tornamos a malha fina mais densa, fazendo \textit{h} se aproximar mais de zero. $h_x = h_y = h \rightarrow 0$) .
  A depender dos valores iniciais de $n_x$ e $n_y$ e os valores das coordenadas do ponto $(x, y)$, claro, \textit{podem haver casos em que o teste falhe}, pois, por exemplo, o ponto $(x, y)$
  pode estar mais perto de um ponto da malha ``fina'' que o mesmo ponto $(x, y)$ da malha ``grossa''. Isso não é esperado, mas pode ocorrer.
\end{enumerate}

\textbf{Consideração importante:} o teste de comportamento de erro irá re-executar o programa com valores superiores para os parâmetros $n_x$ e $n_y$. A depender do seu valor inicial,
a execução do programa pode ficar consideravelmnte demorada, pois o número de pontos na malha para a realização deste teste será \textbf{consideravelmente superior}. A explicação do porque isso é feito
já foi fornecida acima.

\pagebreak

\section{Dedução dos métodos de interpolação: cálculo eficiente dos coeficientes}

\indent\indent Esta seção abordará detalhes do que foi feito na
implementação para resolver o problema do EP2. Todas as definições e suposições
feitas no enunciado já estão sendo consideradas para as explicações que se
seguem. Para a resolução dos problemas do EP3, por favor, pule para a próxima seção.

\subsection{Caso Bilinear}
  \indent\indent Para resolver o problema de interpolação no caso bilinear, precisamos apenas
  considerar os valores da função $f(x, y)$ em cada vértice do \textit{pixel} da
  malha definida pelos 6 parâmetros mandatórios de entrada $n_x, n_y, a_x, b_x, a_y$ e $b_y$.
  Para todos os efeitos, vamos sempre considerar que o ponto $(x_i, y_j)$ refere-se ao vértice
  \textit{inferior esquerdo} do \textit{pixel}, que o ponto $(x_{i + 1}, y_j)$ refere-se ao vértice
  \textit{inferior direito} do \textit{pixel}, que o ponto $(x, y_{j + 1})$ refere-se ao vértice
  \textit{superior esquerdo} do \textit{pixel} e, finalmente, que o ponto $(x_{i + 1}, y_{j + 1})$
  refere-se ao vértice do \textit{pixel} que sobrou, o \textit{superior direito}.

  O polinômio dado no enunciado por

  \begin{center}
    $s_{ij}^L(x, y) = a_{00} + a_{10}(x - x_i) + a_{01}(y - y_j) + a_{11}(x - x_i)(y - y_j)$
  \end{center}

  deve interpolar $f$, e portanto, temos que a condição $f(x, y) \approx s_{ij}^L(x, y)$ deve
  ser satisfeita. Ainda, nos pontos da malha, temos que $f(x, y) = s_{ij}^L(x, y)$, que nos
  leva às condições de interpolação que são fornecidas no enunciado e reescritas abaixo:

  \begin{center}
    $s_{ij}^L(x_i, y_j) = f(x_i, y_j)$ \\
    $s_{ij}^L(x_i, y_{j + 1}) = f(x_i, y_{j + 1})$ \\
    $s_{ij}^L(x_{i + 1}, y_j) = f(x_{i + 1}, y_j)$ \\
    $s_{ij}^L(x_{i + 1}, y_{j + 1}) = f(x_{i + 1}, y_{j + 1})$
  \end{center}

  Que podemos expandi-las para cada um dos quatro vértices do \textit{pixel}:

  \begin{center}
    $s_{ij}^L(x_i, y_j) = a_{00} + a_{10}(x_i - x_i) + a_{01}(y_j - y_j) + a_{11}(x_i - x_i)(y_j - y_j) = f(x_i, y_j)$ \\
    $s_{ij}^L(x_i, y_{j + 1}) = a_{00} + a_{10}(x_i - x_i) + a_{01}(y_{j + 1} - y_j) + a_{11}(x_i - x_i)(y_{j + 1} - y_j) = f(x_i, y_{j + 1})$ \\
    $s_{ij}^L(x_{i + 1}, y_j) = a_{00} + a_{10}(x_{i + 1} - x_i) + a_{01}(y_j - y_j) + a_{11}(x_{i + 1} - x_i)(y_j - y_j) = f(x_{i + 1}, y_j)$ \\
    $s_{ij}^L(x_{i + 1}, y_{j + 1}) = a_{00} + a_{10}(x_{i + 1} - x_i) + a_{01}(y_{j + 1} - y_j) + a_{11}(x_{i + 1} - x_i)(y_{j + 1} - y_j) = f(x_{i + 1}, y_{j + 1})$
  \end{center}

  E podemos reescrever o que está representado acima matricialmente por:

  \begin{center}
    $ \begin{pmatrix}
      a_{00} & 0         & 0         & 0 \\
      a_{00} & 0         & a_{01}h_y & 0 \\
      a_{00} & a_{10}h_x & 0         & 0 \\
      a_{00} & a_{10}h_x & a_{01}h_y & a_{11}h_xh_y
    \end{pmatrix}
    =
    \begin{pmatrix}
      f(x_i, y_j)  \\
      f(x_i, y_{j + 1})  \\
      f(x_{i + 1}, y_j)  \\
      f(x_{i + 1}, y_{j + 1})
    \end{pmatrix}$
  \end{center}

  E, finalmente, obtemos uma maneira de calcular os coeficientes eficientemente:

  \begin{center}
    $ \begin{pmatrix}
      1 & 0   & 0   & 0 \\
      1 & 0   & h_y & 0 \\
      1 & h_x & 0   & 0 \\
      1 & h_x & h_y & h_xh_y
    \end{pmatrix}
    \begin{pmatrix}
      a_{00}  \\
      a_{10}  \\
      a_{01}  \\
      a_{11}
    \end{pmatrix}
    =
    \begin{pmatrix}
      f(x_i, y_j)  \\
      f(x_i, y_{j + 1})  \\
      f(x_{i + 1}, y_j)  \\
      f(x_{i + 1}, y_{j + 1})
    \end{pmatrix}$
  \end{center}

\subsection{Caso Bicúbico}

  \indent\indent O caso bicúbico compartilha semelhanças com o caso bilinear explicado
  na subseção anterior. A diferença é que agora precisamos usar os valores das derivadas
  de primeira ordem com relação às variáveis $x$ e $y$ de cada vértice do \textit{pixel},
  bem como as derivadas mistas de segunda ordem \textbf{também}, pois temos 12 coeficientes
  a mais a serem determinados. Naturalmente, é suposto que a função admita a existência
  das derivadas parciais com relação às variáveis $x$ e $y$, bem como a existência da
  derivada mista de segunda ordem. Em outras palavras, $f$ deve ser de classe $C^2$.
  \textbf{Observação}: o Exercício Programa \textbf{não tem implementações para realizar tais
  checagens}. Está a cargo do usuário colocar uma função que respeite tais condições.

  Analogo ao processo que fizemos na seção do caso bilinear, conjuntamente com a suposição
  da existência das derivadas parciais já mencionadas, reescrevemos todas as 16 condições de
  interpolação do enunciado abaixo:

  \begin{center}
    $s_{ij}^C(x_i, y_j) = f(x_i, y_j)$ \\
    $s_{ij}^C(x_i, y_{j + 1}) = f(x_i, y_{j + 1})$ \\
    $s_{ij}^C(x_{i + 1}, y_j) = f(x_{i + 1}, y_j)$ \\
    $s_{ij}^C(x_{i + 1}, y_{j + 1}) = f(x_{i + 1}, y_{j + 1})$ \\
    $\frac{\partial s_{ij}^C}{\partial y}(x_i, y_j) = \frac{\partial f}{\partial y}(x_i, y_j)$ \\
    $\frac{\partial s_{ij}^C}{\partial x}(x_i, y_j) = \frac{\partial f}{\partial x}(x_i, y_j)$ \\
    $\frac{\partial s_{ij}^C}{\partial y}(x_i, y_{j + 1}) = \frac{\partial f}{\partial y}(x_i, y_{j + 1})$ \\
    $\frac{\partial s_{ij}^C}{\partial x}(x_i, y_{j + 1}) = \frac{\partial f}{\partial x}(x_i, y_{j + 1})$ \\
    $\frac{\partial s_{ij}^C}{\partial y}(x_{i + 1}, y_j) = \frac{\partial f}{\partial y}(x_{i + 1}, y_j)$ \\
    $\frac{\partial s_{ij}^C}{\partial x}(x_{i + 1}, y_j) = \frac{\partial f}{\partial x}(x_{i + 1}, y_j)$ \\
    $\frac{\partial s_{ij}^C}{\partial y}(x_{i + 1}, y_{j + 1}) = \frac{\partial f}{\partial y}(x_{i + 1}, y_{j + 1})$ \\
    $\frac{\partial s_{ij}^C}{\partial x}(x_{i + 1}, y_{j + 1}) = \frac{\partial f}{\partial x}(x_{i + 1}, y_{j + 1})$ \\
    $\frac{\partial^2 s_{ij}^C}{\partial x \partial y}(x_i, y_j) = \frac{\partial^2 f}{\partial x \partial y}(x_i, y_j)$ \\
    $\frac{\partial^2 s_{ij}^C}{\partial x \partial y}(x_i, y_{j + 1}) = \frac{\partial^2 f}{\partial x \partial y}(x_i, y_{j + 1})$ \\
    $\frac{\partial^2 s_{ij}^C}{\partial x \partial y}(x_{i + 1}, y_j) = \frac{\partial^2 f}{\partial x \partial y}(x_{i + 1}, y_j)$ \\
    $\frac{\partial^2 s_{ij}^C}{\partial x \partial y}(x_{i + 1}, y_{j + 1}) = \frac{\partial^2 f}{\partial x \partial y}(x_{i + 1}, y_{j + 1})$
  \end{center}

  E reescrevamos as condições de interpolação no seguinte formato matricial: \newline

  \hspace*{-2.9cm}
  {\footnotesize
    $\begin{pmatrix}
      s_{ij}^C(x_i, y_j)  & s_{ij}^C(x_i, y_{j + 1})             & \frac{\partial s_{ij}^C}{\partial y}(x_i, y_j) & \frac{\partial s_{ij}^C}{\partial y}(x_i, y_{j + 1}) \\
      s_{ij}^C(x_{i + 1}, y_j)  & s_{ij}^C(x_{i + 1}, y_{j + 1}) & \frac{\partial s_{ij}^C}{\partial y}(x_{i + 1}, y_j) & \frac{\partial s_{ij}^C}{\partial y}(x_{i + 1}, y_{j + 1}) \\
      \frac{\partial s_{ij}^C}{\partial x}(x_i, y_j) & \frac{\partial s_{ij}^C}{\partial x}(x_i, y_{j + 1}) & \frac{\partial^2 s_{ij}^C}{\partial x \partial y}(x_i, y_j) & \frac{\partial^2 s_{ij}^C}{\partial x \partial y}(x_i, y_{j + 1}) \\
      \frac{\partial s_{ij}^C}{\partial x}(x_{i + 1}, y_j)  & \frac{\partial s_{ij}^C}{\partial x}(x_{i + 1}, y_{j + 1}) & \frac{\partial^2 s_{ij}^C}{\partial x \partial y}(x_{i + 1}, y_j) & \frac{\partial^2 s_{ij}^C}{\partial x \partial y}(x_{i + 1}, y_{j + 1})
    \end{pmatrix} =
    \begin{pmatrix}
      f(x_i, y_j)                                   & f(x_i, y_{j + 1})                                   & \frac{\partial f}{\partial y}(x_i, y_j)                    & \frac{\partial f}{\partial y}(x_i, y_{j + 1}) \\
      f(x_{i + 1}, y_j)                             & f(x_{i + 1}, y_{j + 1})                             & \frac{\partial f}{\partial y}(x_{i + 1}, y_j)              & \frac{\partial f}{\partial y}(x_{i + 1}, y_{j + 1}) \\
      \frac{\partial f}{\partial x}(x_i, y_j)       & \frac{\partial f}{\partial x}(x_i, y_{j + 1})       & \frac{\partial^2 f}{\partial x \partial y}(x_i, y_j)       & \frac{\partial^2 f}{\partial x \partial y}(x_i, y_{j + 1}) \\
      \frac{\partial f}{\partial x}(x_{i + 1}, y_j) & \frac{\partial f}{\partial x}(x_{i + 1}, y_{j + 1}) & \frac{\partial^2 f}{\partial x \partial y}(x_{i + 1}, y_j) & \frac{\partial^2 f}{\partial x \partial y}(x_{i + 1}, y_{j + 1})
    \end{pmatrix}$}
    \newline

  Consideremos, então, a forma matricial do polinômio $s_{ij}^C(x, y)$ fornecida pelo enunciado, reproduzida abaixo:

  \begin{center}
    $ s_{ij}^C(x, y) = \begin{pmatrix}
      1 & (x - x_i) & (x - x_i)^2 & (x - x_i)^3
    \end{pmatrix}
    \begin{pmatrix}
      a_{00}  & a_{01} & a_{02} & a_{03} \\
      a_{10}  & a_{11} & a_{12} & a_{13} \\
      a_{20}  & a_{21} & a_{22} & a_{23} \\
      a_{30}  & a_{31} & a_{32} & a_{33}
    \end{pmatrix}
    \begin{pmatrix}
      1  \\
      (y - y_j) \\
      (y - y_j)^2  \\
      (y - y_j)^3
    \end{pmatrix}$
  \end{center}

  Temos, finalmente, para cada um dos 16 polinômios $s_{ij}^C(x, y)$:

  \begin{enumerate}[label=\textbf{\arabic*)}]
    \item $s_{ij}^C(x_i, y_j)$

    \begin{flushleft}

      $= \begin{pmatrix}
        1 & (x_i - x_i) & (x_i - x_i)^2 & (x_i - x_i)^3
      \end{pmatrix}
      \begin{pmatrix}
        a_{00}  & a_{01} & a_{02} & a_{03} \\
        a_{10}  & a_{11} & a_{12} & a_{13} \\
        a_{20}  & a_{21} & a_{22} & a_{23} \\
        a_{30}  & a_{31} & a_{32} & a_{33}
      \end{pmatrix}
      \begin{pmatrix}
        1  \\
        (y_j - y_j) \\
        (y_j - y_j)^2  \\
        (y_j - y_j)^3
      \end{pmatrix}$


      $= \begin{pmatrix}
        1 & 0 & 0 & 0
      \end{pmatrix}
      \begin{pmatrix}
        a_{00}  & a_{01} & a_{02} & a_{03} \\
        a_{10}  & a_{11} & a_{12} & a_{13} \\
        a_{20}  & a_{21} & a_{22} & a_{23} \\
        a_{30}  & a_{31} & a_{32} & a_{33}
      \end{pmatrix}
      \begin{pmatrix}
        1  \\
        0  \\
        0  \\
        0
      \end{pmatrix}$

      $= a_{00}$
    \end{flushleft}

    \item $s_{ij}^C(x_i, y_{j + 1})$

    \begin{flushleft}

      $= \begin{pmatrix}
        1 & (x_i - x_i) & (x_i - x_i)^2 & (x_i - x_i)^3
      \end{pmatrix}
      \begin{pmatrix}
        a_{00}  & a_{01} & a_{02} & a_{03} \\
        a_{10}  & a_{11} & a_{12} & a_{13} \\
        a_{20}  & a_{21} & a_{22} & a_{23} \\
        a_{30}  & a_{31} & a_{32} & a_{33}
      \end{pmatrix}
      \begin{pmatrix}
        1  \\
        (y_{j + 1} - y_j) \\
        (y_{j + 1} - y_j)^2  \\
        (y_{j + 1} - y_j)^3
      \end{pmatrix}$


      $= \begin{pmatrix}
        1 & 0 & 0 & 0
      \end{pmatrix}
      \begin{pmatrix}
        a_{00}  & a_{01} & a_{02} & a_{03} \\
        a_{10}  & a_{11} & a_{12} & a_{13} \\
        a_{20}  & a_{21} & a_{22} & a_{23} \\
        a_{30}  & a_{31} & a_{32} & a_{33}
      \end{pmatrix}
      \begin{pmatrix}
        1  \\
        h_y  \\
        h_y^2  \\
        h_y^3
      \end{pmatrix}$

      $= a_{00}  + a_{01}h_y + a_{02}h_y^2 + a_{03}h_y^3 $
    \end{flushleft}

    \item $s_{ij}^C(x_{i + 1}, y_j)$

    \begin{flushleft}

      $= \begin{pmatrix}
        1 & (x_{i + 1} - x_i) & (x_{i + 1} - x_i)^2 & (x_{i + 1} - x_i)^3
      \end{pmatrix}
      \begin{pmatrix}
        a_{00}  & a_{01} & a_{02} & a_{03} \\
        a_{10}  & a_{11} & a_{12} & a_{13} \\
        a_{20}  & a_{21} & a_{22} & a_{23} \\
        a_{30}  & a_{31} & a_{32} & a_{33}
      \end{pmatrix}
      \begin{pmatrix}
        1  \\
        (y_j - y_j) \\
        (y_j - y_j)^2  \\
        (y_j - y_j)^3
      \end{pmatrix}$


      $= \begin{pmatrix}
        1 & h_x & h_x^2 & h_x^3
      \end{pmatrix}
      \begin{pmatrix}
        a_{00}  & a_{01} & a_{02} & a_{03} \\
        a_{10}  & a_{11} & a_{12} & a_{13} \\
        a_{20}  & a_{21} & a_{22} & a_{23} \\
        a_{30}  & a_{31} & a_{32} & a_{33}
      \end{pmatrix}
      \begin{pmatrix}
        1  \\
        0  \\
        0  \\
        0
      \end{pmatrix}$

      $= a_{00} + a_{10}h_x + a_{20}h_x^2 + a_{30}h_x^3$
    \end{flushleft}

    \item $s_{ij}^C(x_{i + 1}, y_{j + 1})$

    \begin{flushleft}
    $= \begin{pmatrix}
      1 & (x_{i + 1} - x_i) & (x_{i + 1} - x_i)^2 & (x_{i + 1} - x_i)^3
    \end{pmatrix}
    \begin{pmatrix}
      a_{00}  & a_{01} & a_{02} & a_{03} \\
      a_{10}  & a_{11} & a_{12} & a_{13} \\
      a_{20}  & a_{21} & a_{22} & a_{23} \\
      a_{30}  & a_{31} & a_{32} & a_{33}
    \end{pmatrix}
    \begin{pmatrix}
      1  \\
      (y_{j + 1} - y_j) \\
      (y_{j + 1} - y_j)^2  \\
      (y_{j + 1} - y_j)^3
    \end{pmatrix}$


    $= \begin{pmatrix}
      1 & h_x & h_x^2 & h_x^3
    \end{pmatrix}
    \begin{pmatrix}
      a_{00}  & a_{01} & a_{02} & a_{03} \\
      a_{10}  & a_{11} & a_{12} & a_{13} \\
      a_{20}  & a_{21} & a_{22} & a_{23} \\
      a_{30}  & a_{31} & a_{32} & a_{33}
    \end{pmatrix}
    \begin{pmatrix}
      1  \\
      h_y  \\
      h_y^2  \\
      h_y^3
    \end{pmatrix}$

    $= a_{00} + a_{01}h_y + a_{02}h_y^2 + a_{03}h_y^3 + a_{10}h_x + a_{11}h_xh_y + a_{12}h_xh_y^2 + a_{13}h_xh_y^3 + a_{20}h_x^2 + a_{21}h_x^2h_y + a_{22}h_x^2h_y^2 + a_{23}h_x^2h_y^3 + a_{30}h_x^3 + a_{31}h_x^3h_y + a_{32}h_x^3h_y^2 + a_{33}h_x^3h_y^3$
    \end{flushleft}

    \item $\frac{\partial s_{ij}^C}{\partial y}(x_i, y_j)$

    \begin{flushleft}
    $= \begin{pmatrix}
      1 & (x_i - x_i) & (x_i - x_i)^2 & (x_i - x_i)^3
    \end{pmatrix}
    \begin{pmatrix}
      a_{00}  & a_{01} & a_{02} & a_{03} \\
      a_{10}  & a_{11} & a_{12} & a_{13} \\
      a_{20}  & a_{21} & a_{22} & a_{23} \\
      a_{30}  & a_{31} & a_{32} & a_{33}
    \end{pmatrix}
    \begin{pmatrix}
      0  \\
      1 \\
      2(y_j - y_j)  \\
      3(y_j - y_j)^2
    \end{pmatrix}$


    $= \begin{pmatrix}
      1 & 0 & 0 & 0
    \end{pmatrix}
    \begin{pmatrix}
      a_{00}  & a_{01} & a_{02} & a_{03} \\
      a_{10}  & a_{11} & a_{12} & a_{13} \\
      a_{20}  & a_{21} & a_{22} & a_{23} \\
      a_{30}  & a_{31} & a_{32} & a_{33}
    \end{pmatrix}
    \begin{pmatrix}
      0  \\
      1  \\
      0  \\
      0
    \end{pmatrix}$

    $= a_{01}$
    \end{flushleft}


    \item $\frac{\partial s_{ij}^C}{\partial y}(x_i, y_{j + 1})$

    \begin{flushleft}
    $= \begin{pmatrix}
      1 & (x_i - x_i) & (x_i - x_i)^2 & (x_i - x_i)^3
    \end{pmatrix}
    \begin{pmatrix}
      a_{00}  & a_{01} & a_{02} & a_{03} \\
      a_{10}  & a_{11} & a_{12} & a_{13} \\
      a_{20}  & a_{21} & a_{22} & a_{23} \\
      a_{30}  & a_{31} & a_{32} & a_{33}
    \end{pmatrix}
    \begin{pmatrix}
      0  \\
      1 \\
      2(y_{j + 1} - y_j)  \\
      3(y_{j + 1} - y_j)^2
    \end{pmatrix}$


    $= \begin{pmatrix}
      1 & 0 & 0 & 0
    \end{pmatrix}
    \begin{pmatrix}
      a_{00}  & a_{01} & a_{02} & a_{03} \\
      a_{10}  & a_{11} & a_{12} & a_{13} \\
      a_{20}  & a_{21} & a_{22} & a_{23} \\
      a_{30}  & a_{31} & a_{32} & a_{33}
    \end{pmatrix}
    \begin{pmatrix}
      0  \\
      1  \\
      2h_y  \\
      3h_y^2
    \end{pmatrix}$

    $= a_{01} + 2a_{02}h_y + 3a_{03}h_y^2$
    \end{flushleft}

    \item $\frac{\partial s_{ij}^C}{\partial y}(x_{i + 1}, y_j)$

    \begin{flushleft}
    $= \begin{pmatrix}
      1 & (x_{i + 1} - x_i) & (x_{i + 1} - x_i)^2 & (x_{i + 1} - x_i)^3
    \end{pmatrix}
    \begin{pmatrix}
      a_{00}  & a_{01} & a_{02} & a_{03} \\
      a_{10}  & a_{11} & a_{12} & a_{13} \\
      a_{20}  & a_{21} & a_{22} & a_{23} \\
      a_{30}  & a_{31} & a_{32} & a_{33}
    \end{pmatrix}
    \begin{pmatrix}
      0  \\
      1 \\
      2(y_j - y_j)  \\
      3(y_j - y_j)^2
    \end{pmatrix}$


    $= \begin{pmatrix}
      1 & h_x & h_x^2 & h_x^3
    \end{pmatrix}
    \begin{pmatrix}
      a_{00}  & a_{01} & a_{02} & a_{03} \\
      a_{10}  & a_{11} & a_{12} & a_{13} \\
      a_{20}  & a_{21} & a_{22} & a_{23} \\
      a_{30}  & a_{31} & a_{32} & a_{33}
    \end{pmatrix}
    \begin{pmatrix}
      0  \\
      1  \\
      0  \\
      0
    \end{pmatrix}$

    $= a_{01} + a_{11}h_x + a_{21}h_x^2 + a_{31}h_x^3$
    \end{flushleft}

    \item $\frac{\partial s_{ij}^C}{\partial y}(x_{i + 1}, y_{j + 1})$

    \begin{flushleft}
    $= \begin{pmatrix}
      1 & (x_{i + 1} - x_i) & (x_{i + 1} - x_i)^2 & (x_{i + 1} - x_i)^3
    \end{pmatrix}
    \begin{pmatrix}
      a_{00}  & a_{01} & a_{02} & a_{03} \\
      a_{10}  & a_{11} & a_{12} & a_{13} \\
      a_{20}  & a_{21} & a_{22} & a_{23} \\
      a_{30}  & a_{31} & a_{32} & a_{33}
    \end{pmatrix}
    \begin{pmatrix}
      0  \\
      1 \\
      2(y_{j + 1} - y_j)  \\
      3(y_{j + 1} - y_j)^2
    \end{pmatrix}$


    $= \begin{pmatrix}
      1 & h_x & h_x^2 & h_x^3
    \end{pmatrix}
    \begin{pmatrix}
      a_{00}  & a_{01} & a_{02} & a_{03} \\
      a_{10}  & a_{11} & a_{12} & a_{13} \\
      a_{20}  & a_{21} & a_{22} & a_{23} \\
      a_{30}  & a_{31} & a_{32} & a_{33}
    \end{pmatrix}
    \begin{pmatrix}
      0  \\
      1  \\
      2h_y  \\
      3h_y^2
    \end{pmatrix}$

    $=  a_{01} + a_{11}h_x + a_{21}h_x^2 + a_{31}h_x^3 + 2(a_{02}h_y + a_{12}h_xh_y + a_{22}h_x^2h_y + a_{32}h_x^3h_y) + 3(a_{03}h_y^2 + a_{13}h_xh_y^2 + a_{23}h_x^2h_y^2 + 3a_{33}h_x^3h_y^2)$
    \end{flushleft}

    \item $\frac{\partial s_{ij}^C}{\partial x}(x_i, y_j)$

    \begin{flushleft}
    $= \begin{pmatrix}
      0 & 1 & 2(x_i - x_i) & 3(x_i - x_i)^2
    \end{pmatrix}
    \begin{pmatrix}
      a_{00}  & a_{01} & a_{02} & a_{03} \\
      a_{10}  & a_{11} & a_{12} & a_{13} \\
      a_{20}  & a_{21} & a_{22} & a_{23} \\
      a_{30}  & a_{31} & a_{32} & a_{33}
    \end{pmatrix}
    \begin{pmatrix}
      1  \\
      (y_j - y_j) \\
      (y_j - y_j)^2  \\
      (y_j - y_j)^3
    \end{pmatrix}$


    $= \begin{pmatrix}
      0 & 1 & 0 & 0
    \end{pmatrix}
    \begin{pmatrix}
      a_{00}  & a_{01} & a_{02} & a_{03} \\
      a_{10}  & a_{11} & a_{12} & a_{13} \\
      a_{20}  & a_{21} & a_{22} & a_{23} \\
      a_{30}  & a_{31} & a_{32} & a_{33}
    \end{pmatrix}
    \begin{pmatrix}
      1  \\
      0  \\
      0  \\
      0
    \end{pmatrix}$

    $= a_{10}$
    \end{flushleft}

    \item $\frac{\partial s_{ij}^C}{\partial x}(x_i, y_{j + 1})$

    \begin{flushleft}
    $= \begin{pmatrix}
      0 & 1 & 2(x_i - x_i) & 3(x_i - x_i)^2
    \end{pmatrix}
    \begin{pmatrix}
      a_{00}  & a_{01} & a_{02} & a_{03} \\
      a_{10}  & a_{11} & a_{12} & a_{13} \\
      a_{20}  & a_{21} & a_{22} & a_{23} \\
      a_{30}  & a_{31} & a_{32} & a_{33}
    \end{pmatrix}
    \begin{pmatrix}
      1  \\
      (y_{j + 1} - y_j) \\
      (y_{j + 1} - y_j)^2  \\
      (y_{j + 1} - y_j)^3
    \end{pmatrix}$


    $= \begin{pmatrix}
      0 & 1 & 0 & 0
    \end{pmatrix}
    \begin{pmatrix}
      a_{00}  & a_{01} & a_{02} & a_{03} \\
      a_{10}  & a_{11} & a_{12} & a_{13} \\
      a_{20}  & a_{21} & a_{22} & a_{23} \\
      a_{30}  & a_{31} & a_{32} & a_{33}
    \end{pmatrix}
    \begin{pmatrix}
      1  \\
      h_y  \\
      h_y^2  \\
      h_y^3
    \end{pmatrix}$

    $= a_{10} + a_{11}h_y + a_{12}h_y^2 + a_{13}h_y^3$
    \end{flushleft}

    \item $\frac{\partial s_{ij}^C}{\partial x}(x_{i + 1}, y_j)$

    \begin{flushleft}
    $= \begin{pmatrix}
      0 & 1 & 2(x_{i + 1} - x_i) & 3(x_{i + 1} - x_i)^2
    \end{pmatrix}
    \begin{pmatrix}
      a_{00}  & a_{01} & a_{02} & a_{03} \\
      a_{10}  & a_{11} & a_{12} & a_{13} \\
      a_{20}  & a_{21} & a_{22} & a_{23} \\
      a_{30}  & a_{31} & a_{32} & a_{33}
    \end{pmatrix}
    \begin{pmatrix}
      1  \\
      (y_j - y_j) \\
      (y_j - y_j)^2  \\
      (y_j - y_j)^3
    \end{pmatrix}$


    $= \begin{pmatrix}
      0 & 1 & 2h_x & 3h_x^2
    \end{pmatrix}
    \begin{pmatrix}
      a_{00}  & a_{01} & a_{02} & a_{03} \\
      a_{10}  & a_{11} & a_{12} & a_{13} \\
      a_{20}  & a_{21} & a_{22} & a_{23} \\
      a_{30}  & a_{31} & a_{32} & a_{33}
    \end{pmatrix}
    \begin{pmatrix}
      1  \\
      0  \\
      0  \\
      0
    \end{pmatrix}$

    $= a_{10} + 2a_{20}h_x + 3a_{30}h_x^2$
    \end{flushleft}

    \item $\frac{\partial s_{ij}^C}{\partial x}(x_{i + 1}, y_{j + 1})$

    \begin{flushleft}
    $= \begin{pmatrix}
      0 & 1 & 2(x_{i + 1} - x_i) & 3(x_{i + 1} - x_i)^2
    \end{pmatrix}
    \begin{pmatrix}
      a_{00}  & a_{01} & a_{02} & a_{03} \\
      a_{10}  & a_{11} & a_{12} & a_{13} \\
      a_{20}  & a_{21} & a_{22} & a_{23} \\
      a_{30}  & a_{31} & a_{32} & a_{33}
    \end{pmatrix}
    \begin{pmatrix}
      1  \\
      (y_{j + 1} - y_j) \\
      (y_{j + 1} - y_j)^2  \\
      (y_{j + 1} - y_j)^3
    \end{pmatrix}$


    $= \begin{pmatrix}
      0 & 1 & 2h_x & 3h_x^2
    \end{pmatrix}
    \begin{pmatrix}
      a_{00}  & a_{01} & a_{02} & a_{03} \\
      a_{10}  & a_{11} & a_{12} & a_{13} \\
      a_{20}  & a_{21} & a_{22} & a_{23} \\
      a_{30}  & a_{31} & a_{32} & a_{33}
    \end{pmatrix}
    \begin{pmatrix}
      1  \\
      h_y  \\
      h_y^2  \\
      h_y^3
    \end{pmatrix}$

    $= a_{10} + a_{11}h_y + a_{12}h_y^2 + a_{13}h_y^3 + 2(a_{20}h_x + a_{21}h_xh_y + a_{22}h_xh_y^2 + a_{23}h_xh_y^3) + 3(a_{30}h_x^2 + a_{31}h_x^2h_y + a_{32}h_x^2h_y^2 + a_{33}h_x^2h_y^3)$
    \end{flushleft}

    \item $\frac{\partial^2 s_{ij}^C}{\partial x \partial y}(x_i, y_j)$

    \begin{flushleft}
    $= \begin{pmatrix}
      0 & 1 & 2(x_i - x_i) & 3(x_i - x_i)^2
    \end{pmatrix}
    \begin{pmatrix}
      a_{00}  & a_{01} & a_{02} & a_{03} \\
      a_{10}  & a_{11} & a_{12} & a_{13} \\
      a_{20}  & a_{21} & a_{22} & a_{23} \\
      a_{30}  & a_{31} & a_{32} & a_{33}
    \end{pmatrix}
    \begin{pmatrix}
      0  \\
      1 \\
      2(y_j - y_j)  \\
      3(y_j - y_j)^2
    \end{pmatrix}$


    $= \begin{pmatrix}
      0 & 1 & 0 & 0
    \end{pmatrix}
    \begin{pmatrix}
      a_{00}  & a_{01} & a_{02} & a_{03} \\
      a_{10}  & a_{11} & a_{12} & a_{13} \\
      a_{20}  & a_{21} & a_{22} & a_{23} \\
      a_{30}  & a_{31} & a_{32} & a_{33}
    \end{pmatrix}
    \begin{pmatrix}
      0  \\
      1  \\
      0  \\
      0
    \end{pmatrix}$

    $= a_{11}$
    \end{flushleft}

    \item $\frac{\partial^2 s_{ij}^C}{\partial x \partial y}(x_i, y_{j + 1})$

    \begin{flushleft}
    $= \begin{pmatrix}
      0 & 1 & 2(x_i - x_i) & 3(x_i - x_i)^2
    \end{pmatrix}
    \begin{pmatrix}
      a_{00}  & a_{01} & a_{02} & a_{03} \\
      a_{10}  & a_{11} & a_{12} & a_{13} \\
      a_{20}  & a_{21} & a_{22} & a_{23} \\
      a_{30}  & a_{31} & a_{32} & a_{33}
    \end{pmatrix}
    \begin{pmatrix}
      0  \\
      1 \\
      2(y_{j + 1} - y_j)  \\
      3(y_{j + 1} - y_j)^2
    \end{pmatrix}$


    $= \begin{pmatrix}
      0 & 1 & 0 & 0
    \end{pmatrix}
    \begin{pmatrix}
      a_{00}  & a_{01} & a_{02} & a_{03} \\
      a_{10}  & a_{11} & a_{12} & a_{13} \\
      a_{20}  & a_{21} & a_{22} & a_{23} \\
      a_{30}  & a_{31} & a_{32} & a_{33}
    \end{pmatrix}
    \begin{pmatrix}
      0  \\
      1  \\
      2h_y  \\
      3h_y^2
    \end{pmatrix}$

    $= a_{11} + 2a_{12}h_y + 3a_{13}h_y^2$
    \end{flushleft}

    \item $\frac{\partial^2 s_{ij}^C}{\partial x \partial y}(x_{i + 1}, y_j)$

    \begin{flushleft}
    $= \begin{pmatrix}
      0 & 1 & 2(x_{i + 1} - x_i) & 3(x_{i + 1} - x_i)^2
    \end{pmatrix}
    \begin{pmatrix}
      a_{00}  & a_{01} & a_{02} & a_{03} \\
      a_{10}  & a_{11} & a_{12} & a_{13} \\
      a_{20}  & a_{21} & a_{22} & a_{23} \\
      a_{30}  & a_{31} & a_{32} & a_{33}
    \end{pmatrix}
    \begin{pmatrix}
      0  \\
      1 \\
      2(y_j - y_j)  \\
      3(y_j - y_j)^2
    \end{pmatrix}$


    $= \begin{pmatrix}
      0 & 1 & 2h_x & 3h_x^2
    \end{pmatrix}
    \begin{pmatrix}
      a_{00}  & a_{01} & a_{02} & a_{03} \\
      a_{10}  & a_{11} & a_{12} & a_{13} \\
      a_{20}  & a_{21} & a_{22} & a_{23} \\
      a_{30}  & a_{31} & a_{32} & a_{33}
    \end{pmatrix}
    \begin{pmatrix}
      0  \\
      1  \\
      0  \\
      0
    \end{pmatrix}$

    $= a_{11} + 2a_{21}h_x + 3a_{31}h_x^2$
    \end{flushleft}

    \item $\frac{\partial^2 s_{ij}^C}{\partial x \partial y}(x_{i + 1}, y_{j + 1})$

    \begin{flushleft}
    $= \begin{pmatrix}
      0 & 1 & 2(x_{i + 1} - x_i) & 3(x_{i + 1} - x_i)^2
    \end{pmatrix}
    \begin{pmatrix}
      a_{00}  & a_{01} & a_{02} & a_{03} \\
      a_{10}  & a_{11} & a_{12} & a_{13} \\
      a_{20}  & a_{21} & a_{22} & a_{23} \\
      a_{30}  & a_{31} & a_{32} & a_{33}
    \end{pmatrix}
    \begin{pmatrix}
      0  \\
      1 \\
      2(y_{j + 1} - y_j)  \\
      3(y_{j + 1} - y_j)^2
    \end{pmatrix}$


    $= \begin{pmatrix}
      0 & 1 & 2h_x & 3h_x^2
    \end{pmatrix}
    \begin{pmatrix}
      a_{00}  & a_{01} & a_{02} & a_{03} \\
      a_{10}  & a_{11} & a_{12} & a_{13} \\
      a_{20}  & a_{21} & a_{22} & a_{23} \\
      a_{30}  & a_{31} & a_{32} & a_{33}
    \end{pmatrix}
    \begin{pmatrix}
      0  \\
      1  \\
      2h_y  \\
      3h_y^2
    \end{pmatrix}$

    $= a_{11} + 2a_{21}h_x + 3a_{31}h_x^2 + 2(a_{12}h_y + 2a_{22}h_xh_y + 3a_{32}h_x^2h_y) + 3(a_{13}h_y^2 + 2a_{23}h_xh_y^2 + 3a_{33}h_x^2h_y^2)$
    \end{flushleft}

  \end{enumerate}

  Uma vez com estes 16 valores obtidos, podemos substituir as instâncias de $s_{ij}^C(x, y)$ na nossa matriz e obter a seguinte expressão matricial: \newline

  \hspace*{-2.5cm}
  {\footnotesize
  $\begin{pmatrix}
    1  & 0   & 0     & 0     \\
    1  & h_x & h_x^2 & h_x^3 \\
    0  & 1   & 0     & 0     \\
    0  & 1   & 2h_x  & 3h_x^2
  \end{pmatrix}
  \begin{pmatrix}
    a_{00}  & a_{01} & a_{02} & a_{03} \\
    a_{10}  & a_{11} & a_{12} & a_{13} \\
    a_{20}  & a_{21} & a_{22} & a_{23} \\
    a_{30}  & a_{31} & a_{32} & a_{33}
  \end{pmatrix}
  \begin{pmatrix}
    1  & 1     & 0 & 0 \\
    0  & h_y   & 1 & 1 \\
    0  & h_y^2 & 0 & 2h_y \\
    0  & h_y^3 & 0 & 3h_y^2
  \end{pmatrix} =
  \begin{pmatrix}
    f(x_i, y_j)                                   & f(x_i, y_{j + 1})                                   & \frac{\partial f}{\partial y}(x_i, y_j)                    & \frac{\partial f}{\partial y}(x_i, y_{j + 1}) \\
    f(x_{i + 1}, y_j)                             & f(x_{i + 1}, y_{j + 1})                             & \frac{\partial f}{\partial y}(x_{i + 1}, y_j)              & \frac{\partial f}{\partial y}(x_{i + 1}, y_{j + 1}) \\
    \frac{\partial f}{\partial x}(x_i, y_j)       & \frac{\partial f}{\partial x}(x_i, y_{j + 1})       & \frac{\partial^2 f}{\partial x \partial y}(x_i, y_j)       & \frac{\partial^2 f}{\partial x \partial y}(x_i, y_{j + 1}) \\
    \frac{\partial f}{\partial x}(x_{i + 1}, y_j) & \frac{\partial f}{\partial x}(x_{i + 1}, y_{j + 1}) & \frac{\partial^2 f}{\partial x \partial y}(x_{i + 1}, y_j) & \frac{\partial^2 f}{\partial x \partial y}(x_{i + 1}, y_{j + 1})
  \end{pmatrix}$}\newline

  Precisamos isolar os coeficientes. Finalmente, supondo que a matrizes que estão multiplicando a matriz de coeficientes $a_{ij}$ sejam invertíveis
  (como uma é transposta da outra, apesar dos termos $h_x$ e $h_y$, basta que apenas uma seja invertível que a outra também será, então a rigor fazemos
  apenas uma suposição, e não duas), temos que:

  \hspace*{-2.5cm}
  {\footnotesize
  $\begin{pmatrix}
    a_{00}  & a_{01} & a_{02} & a_{03} \\
    a_{10}  & a_{11} & a_{12} & a_{13} \\
    a_{20}  & a_{21} & a_{22} & a_{23} \\
    a_{30}  & a_{31} & a_{32} & a_{33}
  \end{pmatrix} =
  \begin{pmatrix}
    1  & 0   & 0     & 0     \\
    0  & 0   & 1     & 0 \\
    \frac{-3}{h_x^2}  & \frac{3}{h_x^2}   & \frac{-2}{h_x}     & \frac{-1}{h_x}    \\
    \frac{2}{h_x^3}  & \frac{-2}{h_x^3}   & \frac{1}{h_x^2}  & \frac{1}{h_x^2}
  \end{pmatrix}
  \begin{pmatrix}
    f(x_i, y_j)                                   & f(x_i, y_{j + 1})                                   & \frac{\partial f}{\partial y}(x_i, y_j)                    & \frac{\partial f}{\partial y}(x_i, y_{j + 1}) \\
    f(x_{i + 1}, y_j)                             & f(x_{i + 1}, y_{j + 1})                             & \frac{\partial f}{\partial y}(x_{i + 1}, y_j)              & \frac{\partial f}{\partial y}(x_{i + 1}, y_{j + 1}) \\
    \frac{\partial f}{\partial x}(x_i, y_j)       & \frac{\partial f}{\partial x}(x_i, y_{j + 1})       & \frac{\partial^2 f}{\partial x \partial y}(x_i, y_j)       & \frac{\partial^2 f}{\partial x \partial y}(x_i, y_{j + 1}) \\
    \frac{\partial f}{\partial x}(x_{i + 1}, y_j) & \frac{\partial f}{\partial x}(x_{i + 1}, y_{j + 1}) & \frac{\partial^2 f}{\partial x \partial y}(x_{i + 1}, y_j) & \frac{\partial^2 f}{\partial x \partial y}(x_{i + 1}, y_{j + 1})
  \end{pmatrix}
  \begin{pmatrix}
    1  & 0 & \frac{-3}{h_y^2} & \frac{2}{h_y^3} \\
    0  & 0 & \frac{3}{h_y^2} & \frac{-2}{h_y^3} \\
    0  & 1 & \frac{-2}{h_y} & \frac{1}{h_y^2} \\
    0  & 0 & \frac{-1}{h_y} & \frac{1}{h_y^2}
  \end{pmatrix}$}\newline

  Uma maneira eficiente de calcular os coeficientes para realizarmos o método da interpolação bicúbica.

  \pagebreak

  \section{Dedução das derivadas parciais em x, em y e mistas.}

  \indent\indent Neste capítulos deduziremos formas de calcular as derivadas parciais  $\frac{\partial f}{\partial x}(x, y)$ e $\frac{\partial f}{\partial y}(x, y)$ e a derivada parcial mista
  $\frac{\partial^2 f}{\partial x \partial y}(x, y)$ a partir da expansão do polinômio de Taylor em torno do ponto $(x_i, y_j)$. Para tal, serão usadas as expansões de 3 pontos: centrada, para frente, para trás,
  para cima e para baixo, a depender do caso em que o ponto $(x_i, y_j)$ está em alguma borda da imagem ou não.

  \subsection{Expansão do polinômio de Taylor para funções de duas variáveis}

  \indent\indent Conforme mencionado anteriormente, será usada a expansão do polinômio de Taylor para função de duas variáveis para aproximar as derivadas. Abaixo é apresentada a forma
  do polinômio $P_n$ com a expansão em torno do ponto $(x_i, y_j)$.

  \vsp

  \begin{center}
  $P_n(x, y) = f\left(x_i, y_j\right) + \frac{\partial f}{\partial x}\left(x_i, y_j\right)\left(x - x_i\right) +
    \frac{\partial f}{\partial y}\left(x_i, y_j\right)\left(y - y_j\right) + $

  $+ \frac{1}{2}\left(\frac{\partial^2 f}{\partial x^2}\left(x_i, y_j\right)\left(x - x_i\right)^2 +
  2\frac{\partial^2 f}{\partial x \partial y}\left(x_i, y_j\right)\left(x - x_i\right)\left(y - y_j\right) +
  \frac{\partial^2 f}{\partial y^2}\left(x_i, y_j\right)\left(y - y_j\right)^2\right) + $

  $+ \dots + \frac{1}{n!}\sum\limits_{k = 0}^{n} \binom{n}{k} \frac{\partial^n f}{\partial x^{n - k} \partial y^k}\left(x_i, y_j\right)\left(x - x_i\right)^{n - k}\left(y - y_j\right)^k$

\end{center}
  \vsp

  \noindent {\tiny Fonte: https://en.wikipedia.org/wiki/Taylor\_series}

  \subsection{Derivada parcial em x}

  \indent\indent Começaremos com a derivada parcial $\frac{\partial f}{\partial x}(x, y)$, considerando a nossa malha de pontos no plano cartesiano. A depender da posição do ponto  $(x_i, y_j)$ que
  queremos aproximar a derivada parcial na malha, devemos considerar um dos três casos abaixo:


  \begin{enumerate}[label=\textbf{\arabic*.}]
    \item O ponto $(x_i, y_j)$ não pertence a uma borda da imagem.
    \item O ponto $(x_i, y_j)$ pertence à borda esquerda da imagem.
    \item O ponto $(x_i, y_j)$ pertence à borda direita da imagem.
  \end{enumerate}

  Como estamos considerando $\frac{\partial f}{\partial x}(x, y)$ no plano cartesiano, podemos desconsiderar os casos do ponto pertencer às bordas superior ou inferior da imagem. A seguir, demonstraremos os três itens
  enumerados acima.

  \pagebreak

  \subsubsection{Caso 1: o ponto não pertence a uma borda}

  \indent\indent Usando o ponto $(x, y_j)$ pois o ``deslocamento'' é exclusivamente horizontal, temos:

  \begin{enumerate}[label=\textbf{\Roman*)}]
    \item $f(x + h_x, y_j) = f(x, y_j) + h_x\left(\frac{\partial f}{\partial x}\left(x, y_j \right)\left(x - x_i \right) + \frac{\partial f}{\partial y}\left(x, y_j \right)\left(y_j - y_j \right) \right) +$

    $+ \frac{h_x^2}{2}\left(\frac{\partial^2 f}{\partial x^2}\left(x, y_j\right)\left(x - x_i\right)^2 + 2\frac{\partial^2 f}{\partial x \partial y}\left(x, y_j \right)\left(x - x_i\right)\left(y_j - y_j\right) +
    \frac{\partial^2 f}{\partial y^2}\left(x, y_j\right)\left(y_j - y_j\right)^2 \right) + O(h_x^3)$

    $= f(x, y_j) + h_x\frac{\partial f}{\partial x}\left(x, y_j \right)\left(x - x_i \right) + \frac{h_x^2}{2}\frac{\partial^2 f}{\partial x^2}\left(x, y_j\right)\left(x - x_i\right)^2 + O(h_x^3)$


    \item $f(x - h_x, y_j) = f(x, y_j) - h_x\left(\frac{\partial f}{\partial x}\left(x, y_j \right)\left(x - x_i \right) + \frac{\partial f}{\partial y}\left(x, y_j \right)\left(y_j - y_j \right) \right) +$

    $+ \frac{h_x^2}{2}\left(\frac{\partial^2 f}{\partial x^2}\left(x, y_j\right)\left(x - x_i\right)^2 + 2\frac{\partial^2 f}{\partial x \partial y}\left(x, y_j \right)\left(x - x_i\right)\left(y_j - y_j\right) +
    \frac{\partial^2 f}{\partial y^2}\left(x, y_j\right)\left(y_j - y_j\right)^2 \right) + O(h_x^3)$

    $= f(x, y_j) - h_x\frac{\partial f}{\partial x}\left(x, y_j \right)\left(x - x_i \right) + \frac{h_x^2}{2}\frac{\partial^2 f}{\partial x^2}\left(x, y_j\right)\left(x - x_i\right)^2 + O(h_x^3)$
  \end{enumerate}

  e de \textbf{I - II} segue que:

  $f(x + h_x, y_j) - f(x - h_x, y_j) = 2h_x\frac{\partial f}{\partial x}\left(x, y_j \right)\left(x - x_i \right) + O(h_x^3)$

  e portanto:

  $\frac{\partial f}{\partial x}\left(x, y_j \right)\left(x - x_i \right) = \frac{f(x + h_x, y_j) - f(x - h_x, y_j)}{2h_x}  + O(h_x^2)$

  \subsubsection{Caso 2: o ponto pertence à borda esquerda}

  \indent\indent Usando o ponto $(x, y_j)$ pois o ``deslocamento'' é exclusivamente horizontal, temos:

  \begin{enumerate}[label=\textbf{\Roman*)}]
    \item $f(x + h_x, y_j) = f(x, y_j) + h_x\left(\frac{\partial f}{\partial x}\left(x, y_j \right)\left(x - x_i \right) + \frac{\partial f}{\partial y}\left(x, y_j \right)\left(y_j - y_j \right) \right) +$

    $+ \frac{h_x^2}{2}\left(\frac{\partial^2 f}{\partial x^2}\left(x, y_j\right)\left(x - x_i\right)^2 + 2\frac{\partial^2 f}{\partial x \partial y}\left(x, y_j \right)\left(x - x_i\right)\left(y_j - y_j\right) +
    \frac{\partial^2 f}{\partial y^2}\left(x, y_j\right)\left(y_j - y_j\right)^2 \right) + O(h_x^3)$

    $= f(x, y_j) + h_x\frac{\partial f}{\partial x}\left(x, y_j \right)\left(x - x_i \right) + \frac{h_x^2}{2}\frac{\partial^2 f}{\partial x^2}\left(x, y_j\right)\left(x - x_i\right)^2 + O(h_x^3)$


    \item $f(x + 2h_x, y_j) = f(x, y_j) + 2h_x\left(\frac{\partial f}{\partial x}\left(x, y_j \right)\left(x - x_i \right) + \frac{\partial f}{\partial y}\left(x, y_j \right)\left(y_j - y_j \right) \right)+ $

    $+ 2h_x^2\left(\frac{\partial^2 f}{\partial x^2}\left(x, y_j\right)\left(x - x_i\right)^2 + 2\frac{\partial^2 f}{\partial x \partial y}\left(x, y_j \right)\left(x - x_i\right)\left(y_j - y_j\right) +
    \frac{\partial^2 f}{\partial y^2}\left(x, y_j\right)\left(y_j - y_j\right)^2 \right) + O(h_x^3)$

    $= f(x, y_j) + 2h_x\frac{\partial f}{\partial x}\left(x, y_j \right)\left(x - x_i \right) + 2h_x^2\frac{\partial^2 f}{\partial x^2}\left(x, y_j\right)\left(x - x_i\right)^2 + O(h_x^3)$
  \end{enumerate}

  e de \textbf{4I - II} segue que:

  $4f(x + h_x, y_j) - f(x + 2h_x, y_j) = 3f(x, y_j) + 2h_x\frac{\partial f}{\partial x}\left(x, y_j \right)\left(x - x_i \right) + O(h_x^3)$

  e portanto:

  $\frac{\partial f}{\partial x}\left(x, y_j \right)\left(x - x_i \right) = \frac{-3f(x, y_j) + 4f(x + h_x, y_j) - f(x + 2h_x, y_j)}{2h_x}  + O(h_x^2)$

  \subsubsection{Caso 3: o ponto pertence à borda direita}

  \indent\indent Usando o ponto $(x, y_j)$ pois o ``deslocamento'' é exclusivamente horizontal, temos:

  \begin{enumerate}[label=\textbf{\Roman*)}]
    \item $f(x - h_x, y_j) = f(x, y_j) - h_x\left(\frac{\partial f}{\partial x}\left(x, y_j \right)\left(x - x_i \right) + \frac{\partial f}{\partial y}\left(x, y_j \right)\left(y_j - y_j \right) \right) +$

    $+ \frac{h_x^2}{2}\left(\frac{\partial^2 f}{\partial x^2}\left(x, y_j\right)\left(x - x_i\right)^2 + 2\frac{\partial^2 f}{\partial x \partial y}\left(x, y_j \right)\left(x - x_i\right)\left(y_j - y_j\right) +
    \frac{\partial^2 f}{\partial y^2}\left(x, y_j\right)\left(y_j - y_j\right)^2 \right) + O(h_x^3)$

    $= f(x, y_j) - h_x\frac{\partial f}{\partial x}\left(x, y_j \right)\left(x - x_i \right) + \frac{h_x^2}{2}\frac{\partial^2 f}{\partial x^2}\left(x, y_j\right)\left(x - x_i\right)^2 + O(h_x^3)$


    \item $f(x - 2h_x, y_j) = f(x, y_j) - 2h_x\left(\frac{\partial f}{\partial x}\left(x, y_j \right)\left(x - x_i \right) + \frac{\partial f}{\partial y}\left(x, y_j \right)\left(y_j - y_j \right) \right) +$

    $+ 2h_x^2\left(\frac{\partial^2 f}{\partial x^2}\left(x, y_j\right)\left(x - x_i\right)^2 + 2\frac{\partial^2 f}{\partial x \partial y}\left(x, y_j \right)\left(x - x_i\right)\left(y_j - y_j\right) +
    \frac{\partial^2 f}{\partial y^2}\left(x, y_j\right)\left(y_j - y_j\right)^2 \right) + O(h_x^3)$

    $= f(x, y_j) - 2h_x\frac{\partial f}{\partial x}\left(x, y_j \right)\left(x - x_i \right) + 2h_x^2\frac{\partial^2 f}{\partial x^2}\left(x, y_j\right)\left(x - x_i\right)^2 + O(h_x^3)$
  \end{enumerate}

  e de \textbf{4I - II} segue que:

  $4f(x - h_x, y_j) - f(x - 2h_x, y_j) = 3f(x, y_j) - 2h_x\frac{\partial f}{\partial x}\left(x, y_j \right)\left(x - x_i \right) + O(h_x^3)$

  e portanto:

  $\frac{\partial f}{\partial x}\left(x, y_j \right)\left(x - x_i \right) = \frac{3f(x, y_j) - 4f(x - h_x, y_j) + f(x - 2h_x, y_j)}{2h_x}  + O(h_x^2)$

  \subsection{Derivada parcial em y}


  \indent\indent Tomemos agora a derivada parcial $\frac{\partial f}{\partial y}(x, y)$, considerando a nossa malha de pontos no plano cartesiano. A depender da posição do ponto  $(x_i, y_j)$ que
  queremos aproximar a derivada parcial na malha, devemos considerar um dos três casos abaixo:

  \begin{enumerate}[label=\textbf{\arabic*.}]
    \item O ponto $(x_i, y_j)$ não pertence a uma borda da imagem.
    \item O ponto $(x_i, y_j)$ pertence à borda inferior da imagem.
    \item O ponto $(x_i, y_j)$ pertence à borda superior da imagem.
  \end{enumerate}

  Como estamos considerando $\frac{\partial f}{\partial y}(x, y)$ no plano cartesiano, podemos desconsiderar os casos do ponto pertencer às bordas laterais da imagem. A seguir, demonstraremos os três itens
  enumerados acima.

  \subsubsection{Caso 1: o ponto não pertence a uma borda}

  \indent\indent Usando o ponto $(x_i, y)$ pois o ``deslocamento'' é exclusivamente vertical, temos:

  \begin{enumerate}[label=\textbf{\Roman*)}]
    \item $f(x_i, y + h_y) = f(x_i, y) + h_y\left(\frac{\partial f}{\partial x}\left(x_i, y \right)\left(x_i - x_i \right) + \frac{\partial f}{\partial y}\left(x_i, y \right)\left(y - y_j \right) \right) +$

    $+ \frac{h_y^2}{2}\left(\frac{\partial^2 f}{\partial x^2}\left(x_i, y\right)\left(x_i - x_i\right)^2 + 2\frac{\partial^2 f}{\partial x \partial y}\left(x_i, y \right)\left(x_i - x_i\right)\left(y - y_j\right) +
    \frac{\partial^2 f}{\partial y^2}\left(x_i, y\right)\left(y - y_j\right)^2 \right) + O(h_y^3)$

    $= f(x_i, y) + h_y\frac{\partial f}{\partial y}\left(x_i, y \right)\left(y - y_j \right) + \frac{h_y^2}{2}\frac{\partial^2 f}{\partial y^2}\left(x_i, y\right)\left(y - y_j\right)^2 + O(h_y^3)$


    \item $f(x_i, y - h_y) = f(x_i, y) - h_y\left(\frac{\partial f}{\partial x}\left(x_i, y \right)\left(x_i - x_i \right) + \frac{\partial f}{\partial y}\left(x_i, y \right)\left(y - y_j \right) \right) +$

    $+ \frac{h_y^2}{2}\left(\frac{\partial^2 f}{\partial x^2}\left(x_i, y\right)\left(x_i - x_i\right)^2 + 2\frac{\partial^2 f}{\partial x \partial y}\left(x_i, y \right)\left(x_i - x_i\right)\left(y - y_j\right) +
    \frac{\partial^2 f}{\partial y^2}\left(x_i, y\right)\left(y - y_j\right)^2 \right) + O(h_y^3)$

    $= f(x_i, y) - h_y\frac{\partial f}{\partial y}\left(x_i, y \right)\left(y - y_j \right) + \frac{h_y^2}{2}\frac{\partial^2 f}{\partial y^2}\left(x_i, y\right)\left(y - y_j\right)^2 + O(h_y^3)$
  \end{enumerate}

  e de \textbf{I - II} segue que:

  $f(x_i, y + h_y) - f(x_i, y - h_y) = 2h_y\frac{\partial f}{\partial y}\left(x_i, y \right)\left(y - y_j \right) + O(h_y^3)$

  e portanto:

  $\frac{\partial f}{\partial y}\left(x_i, y \right)\left(y - y_j \right) = \frac{f(x_i, y + h_y) - f(x_i, y - h_y)}{2h_y}  + O(h_y^2)$

  \subsubsection{Caso 2: o ponto pertence à borda inferior}

  \indent\indent Usando o ponto $(x_i, y)$ pois o ``deslocamento'' é exclusivamente vertical, temos:

  \begin{enumerate}[label=\textbf{\Roman*)}]
    \item $f(x_i, y + h_y) = f(x_i, y) + h_y\left(\frac{\partial f}{\partial x}\left(x_i, y \right)\left(x_i - x_i \right) + \frac{\partial f}{\partial y}\left(x_i, y \right)\left(y - y_j \right) \right) +$

    $+ \frac{h_y^2}{2}\left(\frac{\partial^2 f}{\partial x^2}\left(x_i, y\right)\left(x_i - x_i\right)^2 + 2\frac{\partial^2 f}{\partial x \partial y}\left(x_i, y \right)\left(x_i - x_i\right)\left(y - y_j\right) +
    \frac{\partial^2 f}{\partial y^2}\left(x_i, y\right)\left(y - y_j\right)^2 \right) + O(h_y^3)$

    $= f(x_i, y) + h_y\frac{\partial f}{\partial y}\left(x_i, y \right)\left(y - y_j \right) + \frac{h_y^2}{2}\frac{\partial^2 f}{\partial y^2}\left(x_i, y\right)\left(y - y_j\right)^2 + O(h_y^3)$


    \item $f(x_i, y + 2h_y) = f(x_i, y) + 2h_y\left(\frac{\partial f}{\partial x}\left(x_i, y \right)\left(x_i - x_i \right) + \frac{\partial f}{\partial y}\left(x_i, y \right)\left(y - y_j \right) \right) +$

    $+ 2h_y^2\left(\frac{\partial^2 f}{\partial x^2}\left(x_i, y\right)\left(x_i - x_i\right)^2 + 2\frac{\partial^2 f}{\partial x \partial y}\left(x_i, y \right)\left(x_i - x_i\right)\left(y - y_j\right) +
    \frac{\partial^2 f}{\partial y^2}\left(x_i, y\right)\left(y - y_j\right)^2 \right) + O(h_y^3)$

    $= f(x_i, y) + 2h_y\frac{\partial f}{\partial y}\left(x_i, y \right)\left(y - y_j \right) + 2h_y^2\frac{\partial^2 f}{\partial y^2}\left(x_i, y\right)\left(y - y_j\right)^2 + O(h_y^3)$
  \end{enumerate}

  e de \textbf{4I - II} segue que:

  $4f(x_i, y + h_y) - f(x_i, y + 2h_y) = 3f(x_i, y) + 2h_y\frac{\partial f}{\partial y}\left(x_i, y \right)\left(y - y_j \right) + O(h_y^3)$

  e portanto:

  $\frac{\partial f}{\partial y}\left(x_i, y \right)\left(y - y_j \right) = \frac{-3f(x_i, y) + 4f(x_i, y + h_y) - f(x_i, y + 2h_y)}{2h_y}  + O(h_y^2)$


  \subsubsection{Caso 3: o ponto pertence à borda superior}

  \indent\indent Usando o ponto $(x_i, y)$ pois o ``deslocamento'' é exclusivamente vertical, temos:

  \begin{enumerate}[label=\textbf{\Roman*)}]
    \item $f(x_i, y - h_y) = f(x_i, y) - h_y\left(\frac{\partial f}{\partial x}\left(x_i, y \right)\left(x_i - x_i \right) + \frac{\partial f}{\partial y}\left(x_i, y \right)\left(y - y_j \right) \right) +$

    $+ \frac{h_y^2}{2}\left(\frac{\partial^2 f}{\partial x^2}\left(x_i, y\right)\left(x_i - x_i\right)^2 + 2\frac{\partial^2 f}{\partial x \partial y}\left(x_i, y \right)\left(x_i - x_i\right)\left(y - y_j\right) +
    \frac{\partial^2 f}{\partial y^2}\left(x_i, y\right)\left(y - y_j\right)^2 \right) + O(h_y^3)$

    $= f(x_i, y) - h_y\frac{\partial f}{\partial y}\left(x_i, y \right)\left(y - y_j \right) + \frac{h_y^2}{2}\frac{\partial^2 f}{\partial y^2}\left(x_i, y\right)\left(y - y_j\right)^2 + O(h_y^3)$


    \item $f(x_i, y - 2h_y) = f(x_i, y) - 2h_y\left(\frac{\partial f}{\partial x}\left(x_i, y \right)\left(x_i - x_i \right) + \frac{\partial f}{\partial y}\left(x_i, y \right)\left(y - y_j \right) \right) +$

    $+ 2h_y^2\left(\frac{\partial^2 f}{\partial x^2}\left(x_i, y\right)\left(x_i - x_i\right)^2 + 2\frac{\partial^2 f}{\partial x \partial y}\left(x_i, y \right)\left(x_i - x_i\right)\left(y - y_j\right) +
    \frac{\partial^2 f}{\partial y^2}\left(x_i, y\right)\left(y - y_j\right)^2 \right) + O(h_y^3)$

    $= f(x_i, y) - 2h_y\frac{\partial f}{\partial y}\left(x_i, y \right)\left(y - y_j \right) + 2h_y^2\frac{\partial^2 f}{\partial y^2}\left(x_i, y\right)\left(y - y_j\right)^2 + O(h_y^3)$
  \end{enumerate}

  e de \textbf{4I - II} segue que:

  $4f(x_i, y - h_y) - f(x_i, y - 2h_y) = 3f(x_i, y) - 2h_y\frac{\partial f}{\partial y}\left(x_i, y \right)\left(y - y_j \right) + O(h_y^3)$

  e portanto:

  $\frac{\partial f}{\partial y}\left(x_i, y \right)\left(y - y_j \right) = \frac{3f(x_i, y) - 4f(x_i, y - h_y) + f(x_i, y - 2h_y)}{2h_y}  + O(h_y^2)$


  \subsection{Derivada parcial mista}

  \indent\indent Tomemos agora a derivada parcial mista $\frac{\partial^2 f}{\partial x \partial y}(x, y)$, considerando a nossa malha de pontos no plano cartesiano. A depender da posição do ponto  $(x_i, y_j)$ que
  queremos aproximar a derivada parcial na malha, devemos considerar um dos três casos abaixo:


  \begin{enumerate}[label=\textbf{\arabic*.}]
    \item O ponto $(x_i, y_j)$ não pertence a uma borda da imagem.
    \item O ponto $(x_i, y_j)$ pertence à borda esquerda da imagem.
    \item O ponto $(x_i, y_j)$ pertence à borda direita da imagem.
  \end{enumerate}

  Como estamos considerando $\frac{\partial^2 f}{\partial x \partial y}(x, y)$ no plano cartesiano, podemos desconsiderar os casos do ponto pertencer às bordas superior ou inferior da imagem. A seguir, demonstraremos os três itens
  enumerados acima.

  \textbf{Observação:} perceba que estamos calculando a derivada parcial com relação à x na função $\frac{\partial f}{\partial y}(x, y)$, e como isso significa considerar o eixo das abcissas, não é necessário trabalhar os
  casos das bordas superior e inferior.


  \subsubsection{Caso 1: o ponto não pertence a uma borda}

  \indent\indent Usando o ponto $(x, y_j)$ pois o ``deslocamento'' é exclusivamente horizontal, temos:

  \begin{enumerate}[label=\textbf{\Roman*)}]
    \item $\frac{\partial f}{\partial y}(x + h_x, y_j) = \frac{\partial f}{\partial y}(x, y_j) + h_x\left(\frac{\partial^2 f}{\partial x \partial y}\left(x, y_j \right)\left(x - x_i \right) + \frac{\partial^2 f}{\partial y^2}\left(x, y_j \right)\left(y_j - y_j \right) \right) +$

    $+ \frac{h_x^2}{2}\left(\frac{\partial^3 f}{\partial x^2 \partial y}\left(x, y_j\right)\left(x - x_i\right)^2 + 2\frac{\partial^3 f}{\partial x \partial y^2}\left(x, y_j \right)\left(x - x_i\right)\left(y_j - y_j\right) +
    \frac{\partial^3 f}{\partial y^3}\left(x, y_j\right)\left(y_j - y_j\right)^2 \right) + O(h_x^3)$

    $= \frac{\partial f}{\partial y}(x, y_j) + h_x\frac{\partial^2 f}{\partial x \partial y}\left(x, y_j \right)\left(x - x_i \right) + \frac{h_x^2}{2}\frac{\partial^3 f}{\partial x^2 \partial y}\left(x, y_j\right)\left(x - x_i\right)^2 + O(h_x^3)$


    \item $\frac{\partial f}{\partial y}(x - h_x, y_j) = \frac{\partial f}{\partial y}(x, y_j) - h_x\left(\frac{\partial^2 f}{\partial x \partial y}\left(x, y_j \right)\left(x - x_i \right) + \frac{\partial^2 f}{\partial y^2}\left(x, y_j \right)\left(y_j - y_j \right) \right) +$

    $+ \frac{h_x^2}{2}\left(\frac{\partial^3 f}{\partial x^2 \partial y}\left(x, y_j\right)\left(x - x_i\right)^2 + 2\frac{\partial^3 f}{\partial x \partial y^2}\left(x, y_j \right)\left(x - x_i\right)\left(y_j - y_j\right) +
    \frac{\partial^3 f}{\partial y^3}\left(x, y_j\right)\left(y_j - y_j\right)^2 \right) + O(h_x^3)$

    $= \frac{\partial f}{\partial y}(x, y_j) - h_x\frac{\partial^2 f}{\partial x \partial y}\left(x, y_j \right)\left(x - x_i \right) + \frac{h_x^2}{2}\frac{\partial^3 f}{\partial x^2 \partial y}\left(x, y_j\right)\left(x - x_i\right)^2 + O(h_x^3)$
  \end{enumerate}

  e de \textbf{I - II} segue que:

  $\frac{\partial f}{\partial y}(x + h_x, y_j) - \frac{\partial f}{\partial y}(x - h_x, y_j) = 2h_x\frac{\partial^2 f}{\partial x \partial y}\left(x, y_j \right)\left(x - x_i \right) + O(h_x^3)$

  e portanto:

  $\frac{\partial^2 f}{\partial x \partial y}\left(x, y_j \right)\left(x - x_i \right) = \frac{\frac{\partial f}{\partial y}(x + h_x, y_j) - \frac{\partial f}{\partial y}(x - h_x, y_j)}{2h_x}  + O(h_x^2)$

  \subsubsection{Caso 2: o ponto pertence à borda esquerda}

  \indent\indent Usando o ponto $(x, y_j)$ pois o ``deslocamento'' é exclusivamente horizontal, temos:

  \begin{enumerate}[label=\textbf{\Roman*)}]
    \item $\frac{\partial f}{\partial y}(x + h_x, y_j) = \frac{\partial f}{\partial y}(x, y_j) + h_x\left(\frac{\partial^2 f}{\partial x \partial y}\left(x, y_j \right)\left(x - x_i \right) + \frac{\partial^2 f}{\partial y^2}\left(x, y_j \right)\left(y_j - y_j \right) \right) +$

    $+ \frac{h_x^2}{2}\left(\frac{\partial^3 f}{\partial x^2 \partial y}\left(x, y_j\right)\left(x - x_i\right)^2 + 2\frac{\partial^3 f}{\partial x \partial y^2}\left(x, y_j \right)\left(x - x_i\right)\left(y_j - y_j\right) +
    \frac{\partial^3 f}{\partial y^3}\left(x, y_j\right)\left(y_j - y_j\right)^2 \right) + O(h_x^3)$

    $= \frac{\partial f}{\partial y}(x, y_j) + h_x\frac{\partial^2 f}{\partial x \partial y}\left(x, y_j \right)\left(x - x_i \right) + \frac{h_x^2}{2}\frac{\partial^3 f}{\partial x^2 \partial y}\left(x, y_j\right)\left(x - x_i\right)^2 + O(h_x^3)$

    \item $\frac{\partial f}{\partial y}(x + 2h_x, y_j) = \frac{\partial f}{\partial y}(x, y_j) + 2h_x\left(\frac{\partial^2 f}{\partial x \partial y}\left(x, y_j \right)\left(x - x_i \right) + \frac{\partial^2 f}{\partial y^2}\left(x, y_j \right)\left(y_j - y_j \right) \right) +$

    $+ 2h_x^2\left(\frac{\partial^3 f}{\partial x^2 \partial y}\left(x, y_j\right)\left(x - x_i\right)^2 + 2\frac{\partial^3 f}{\partial x \partial y^2}\left(x, y_j \right)\left(x - x_i\right)\left(y_j - y_j\right) +
    \frac{\partial^3 f}{\partial y^3}\left(x, y_j\right)\left(y_j - y_j\right)^2 \right) + O(h_x^3)$

    $= \frac{\partial f}{\partial y}(x, y_j) + 2h_x\frac{\partial^2 f}{\partial x \partial y}\left(x, y_j \right)\left(x - x_i \right) + 2h_x^2\frac{\partial^3 f}{\partial x^2 \partial y}\left(x, y_j\right)\left(x - x_i\right)^2 + O(h_x^3)$
  \end{enumerate}

  e de \textbf{4I - II} segue que:

  $4\frac{\partial f}{\partial y}(x + h_x, y_j) - \frac{\partial f}{\partial y}(x + 2h_x, y_j) = 3\frac{\partial f}{\partial y}(x, y_j) + 2h_x\frac{\partial^2 f}{\partial x \partial y}\left(x, y_j \right)\left(x - x_i \right) + O(h_x^3)$

  e portanto:

  $\frac{\partial^2 f}{\partial x \partial y}\left(x, y_j \right)\left(x - x_i \right) = \frac{-3\frac{\partial f}{\partial y}(x, y_j) + 4\frac{\partial f}{\partial y}(x + h_x, y_j) - \frac{\partial f}{\partial y}(x + 2h_x, y_j)}{2h_x}  + O(h_x^2)$


  \subsubsection{Caso 3: o ponto pertence à borda direita}

  \begin{enumerate}[label=\textbf{\Roman*)}]
    \item $\frac{\partial f}{\partial y}(x - h_x, y_j) = \frac{\partial f}{\partial y}(x, y_j) - h_x\left(\frac{\partial^2 f}{\partial x \partial y}\left(x, y_j \right)\left(x - x_i \right) + \frac{\partial^2 f}{\partial y^2}\left(x, y_j \right)\left(y_j - y_j \right) \right) +$

    $+ \frac{h_x^2}{2}\left(\frac{\partial^3 f}{\partial x^2 \partial y}\left(x, y_j\right)\left(x - x_i\right)^2 + 2\frac{\partial^3 f}{\partial x \partial y^2}\left(x, y_j \right)\left(x - x_i\right)\left(y_j - y_j\right) +
    \frac{\partial^3 f}{\partial y^3}\left(x, y_j\right)\left(y_j - y_j\right)^2 \right) + O(h_x^3)$

    $= \frac{\partial f}{\partial y}(x, y_j) - h_x\frac{\partial^2 f}{\partial x \partial y}\left(x, y_j \right)\left(x - x_i \right) + \frac{h_x^2}{2}\frac{\partial^3 f}{\partial x^2 \partial y}\left(x, y_j\right)\left(x - x_i\right)^2 + O(h_x^3)$

    \item $\frac{\partial f}{\partial y}(x - 2h_x, y_j) = \frac{\partial f}{\partial y}(x, y_j) - 2h_x\left(\frac{\partial^2 f}{\partial x \partial y}\left(x, y_j \right)\left(x - x_i \right) + \frac{\partial^2 f}{\partial y^2}\left(x, y_j \right)\left(y_j - y_j \right) \right) +$

    $+ 2h_x^2\left(\frac{\partial^3 f}{\partial x^2 \partial y}\left(x, y_j\right)\left(x - x_i\right)^2 + 2\frac{\partial^3 f}{\partial x \partial y^2}\left(x, y_j \right)\left(x - x_i\right)\left(y_j - y_j\right) +
    \frac{\partial^3 f}{\partial y^3}\left(x, y_j\right)\left(y_j - y_j\right)^2 \right) + O(h_x^3)$

    $= \frac{\partial f}{\partial y}(x, y_j) - 2h_x\frac{\partial^2 f}{\partial x \partial y}\left(x, y_j \right)\left(x - x_i \right) + 2h_x^2\frac{\partial^3 f}{\partial x^2 \partial y}\left(x, y_j\right)\left(x - x_i\right)^2 + O(h_x^3)$
  \end{enumerate}

  e de \textbf{4I - II} segue que:

  $4\frac{\partial f}{\partial y}(x - h_x, y_j) - \frac{\partial f}{\partial y}(x - 2h_x, y_j) = 3\frac{\partial f}{\partial y}(x, y_j) - 2h_x\frac{\partial^2 f}{\partial x \partial y}\left(x, y_j \right)\left(x - x_i \right) + O(h_x^3)$

  e portanto:

  $\frac{\partial^2 f}{\partial x \partial y}\left(x, y_j \right)\left(x - x_i \right) = \frac{3\frac{\partial f}{\partial y}(x, y_j) - 4\frac{\partial f}{\partial y}(x - h_x, y_j) + \frac{\partial f}{\partial y}(x - 2h_x, y_j)}{2h_x}  + O(h_x^2)$

  \pagebreak

  \section{Considerações Finais (e pessoais)}

  \indent\indent Do EP2 para este, resolvi alguns problemas de falta de conhecimento que tinha com o \textit{GNU-Octave} para oferecer um código mais limpo e eficiente. A qualidade do código
  com relação ao último EP está incomparavelmente superior, e desta vez eu estou entregando o EP satisfeito com os resultados obtidos e a qualidade do código. \textbf{:p}

  Com relação à manipulação de imagens, o problema que eu tive com os índices no EP2 foi aparentemente resolvido neste EP, dado que eu consegui atingir resultados deveras satisfatórios para a interpolação da imagem exemplo.
  Este EP3 não foi feito copiando e colando o EP2 e `melhorando as partes ruins'. Tudo foi replanejado e reimplementado, ora até usando outras estruturas de dados, tal como \textit{Struct Arrays}.

  Ainda, a maneira que eu aproximei as imagens no EP2 não usou nas bordas as expansões de Taylor para 3 pontos para frente, para trás, para cima e para baixo (apenas fazia a média aritmética entre o ponto da borda
  e o \textit{pixel} vizinho conhecido do eixo correspondente), diferentemente do que foi feito neste EP3.

  Finalmente, como o EP3 sugeriu uma idéia de continuidade do EP2 (e pensando em boas práticas de desenvolvimento de \textit{software}), eu resolvi aproveitar a explicação da dedução dos cálculos dos
  coeficientes dos métodos de interpolação nessa documentação apenas porque me pareceu a coisa mais correta a se fazer, mesmo que isso não tivesse sido pedido no enunciado. Com isso a documentação fica, de fato, completa.

\end{document}
